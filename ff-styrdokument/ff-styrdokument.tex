\documentclass{../resources/dgovdoc}


\usepackage[utf8]{inputenc}

\usepackage{hyperref}

\title{Festkommittén}

\begin{document}

\maketitle

\section{Beskrivning}

Festkommiten är en underförening till Fysikums Kamratförening (FysKam) tillika nvf-sektionen vid Uppsala Teknolog- och Naturvetarkår (UTN).

Festkommittén skall följa av dessa (FysKam och UTN) utgivna policys inom applicerbara områden, såsom diskriminering och alkohol.


\section{Syfte}

\subsection{Officiellt syfte}

Festkommittén skall verka för ökad gemenskap bland FysKams medlemmar.

\subsection{Inofficiellt syfte}

Festkommittén bör utöver sitt officiella syfte verka för att engagera nya medlemar så att dessa motiveras att i framtiden engagera sig genom FysKams övriga förtroendeposter såsom FysKams och studierådets styrelse.

\section{Struktur}

Festkommitténs styrelse består av Hufvudansvarig, Vice Hufvudansvarig, Bryggmästare, propagandaminister och Kvittokladdare.

Vid behov kan fler poster tillsättas. Rekommenderade sådana är kvittosamlare, badanka.

Utöver Festkommitténs styrelseposter finns även hederstitlar. Personer som innehar dessa bör vara tidigare styrelsemedlammar, deras syfte är att föra vidare erfarenheter och stötta festkommittén i dess arbete. Dessa personer bör bjudas in till att deltaga på alla möten men har inget krav på sig att utföra något arbete. Exempel på traditionsenliga titlar är Kalaspuff och kalaspingla.

\section{Mötesritualer och liknande}

Dessa punkter bör inte tas som strikta regler utan mer som stämningshöjande riktlinjer. Mötesordförande äga rätt att bortse från dessa regler under mötets gång så länge inga direktiv från FysKam bryts.

Som mötesdeltagande räknas de som är närvarande vid mötet.

Rösträtt ägas av de mötesdeltagande som innehar någon form av dryck.

Röstning sker genom skål, där enkelmajoritet är beslutande.

Mötesdeltagande äga rätt att begära "en kötid" om dryck saknas vid röstning. Röstningen skjuts då upp tills denne haft möjlighet att införskaffa en sådan.

Mötesdeltagande äga rätt att muta övriga mötesdeltagande med dryck. Om mutan godtages övertar mutaren den mutatde personens röst och äga därmed rätt att lägga två röster i nästkommande röstning.

Varje möte skall inledas och avslutas med en skål. Mötets andra skål bör vara en gravskål i forna festares ära.

Protokoll skall föras på kvitton där det köpts OH-dryck, eller vid nödfall, kvitton som det spillts OH-dryck på

Alla i styrelsen äga rätt att spontant öppna möte. Beslut fattade på spontantmöten skall förklaras giltiga eller ogiltiga på nästkommande av Hufvudansvarig utlyst möte.

\section{Ekonomi}

Festkommittén skall föra budget enligt FysKams regi.
Eventuell vinst bör gå till sektionen om den inte anses behövas till framtida eventemang.

\section{Begreppförklaring}

\begin{itemize}
\item Rösträtt - Rätten att rösta, innehavs av alla mötesdeltagande med dryck.
\item Kötid - uppskjutning av röst på mötesdeltagandes uppmaning i syfte att    införskaffa rösträtt.
\item Muta -  Köp av röst från annan mötesdeltagande.
\item Skål - Röstning
\item Festmöte - Festkommitténs möten
\item Årsmöte - Festkommitténs högst beslutande organ.
\item Hufvudansvarig - Ordförande
\item Vice Hufvudansvarig - Vice Ordförande
\item Bryggmästare - Ansvarig för mat och dryck på Festkommitténs aktiviteter.
\item Propagandaminsiter - Information/reklamansvarig.
\item Badanka - Säkerhetsansvarig, ansvarar även för den årliga sektionsklädseltvätten i fyrisån.
\item Kvittokladare - Sekreterare.
\item Kvittosamlare - Ansvarar för att tomma mötesprotokoll finns tillgänliga åt kvittokladdaren
\item SSSSSSSSS - Sångarnas Sångsuppé Som Sker Samtidigt Som Sångerna Sjungs.
\item Kamratskapen - Stor fest, se FysKams reglemente.
\item Kalaspuff/Kalaspingla - Hederstitlar.
\item Hedertitel - Titel som innebär lite arbete och mycket ära.
\item Chatå Skamvrå - FysKams officiella vin, bryggs av bryggmästaren.
\item Glöggus fysikumus - Det felaktiga namnet på FysKams officiella glögg. Bryggs av bryggmästaren.
\end{itemize}

\section{Historia är trevligt}

Då festkommittén älskar historia och att förhöja sig själva i dess ljus bör det till och från genomföras trevliga tillställningar där de som genom åren engagerat sig i festkommittén bjuds in.
På dessa tillställningar anses Chatå Skamvrå vara en utmärkt dryck.

\end{document}
