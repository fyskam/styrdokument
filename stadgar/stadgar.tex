\documentclass{dgovdoc}


\usepackage[utf8]{inputenc}

\usepackage{hyperref}

\title{Stadgar}

\begin{document}

\maketitle

\section{Allmänt}

\subsection{Namn}
\label{sec:namn}

Föreningens namn är Fysikums Kamratförening (FysKam), nedan benämnd sektionen.

\subsection{Syfte}

Sektionens syfte är

\begin{itemize}
  \item att utveckla och upprätthålla kamratskap och sammanhållning bland
    sektionens medlemmar
  \item att skapa och upprätthålla goda kontakter med närstående personer och
    organisationer
  \item att aktivt motverka diskriminering inom sektionen.
\end{itemize}

\subsection{Säte}

Sektionen har sitt säte i Uppsala.

\subsection{Obundenhet \& idealitet}

Sektionen är en politiskt, ideologiskt och religöst obunden ideell förening

\subsection{Verksamhetsår}

Sektionens verksamhetsår löper från 1 januari till 31 december.

\subsection{Styrdokument}
\label{sec:styrdokument}

\subsubsection{Tillgänglighet}

Gällande stadgar och andra styrdokument skall finnas tillgängliga för samtliga
sektionsmedlemmar på sektionens officiella webbplats.

\subsubsection{Stadgetolkning}

Tolkningstvister om dessa stadgar hänskjuts till årsmötet 
eller i brådskande fall till styrelsen. Vid tvist om tolkning av stadgar 
äger styrelsens tolkning företräde tills tvist avgjorts 
på årsmöte eller extra årsmöte. Om tvist uppstått inom 
styrelsen skall berörd fråga bordläggas tills tvist 
avgjorts på årsmöte eller extra årsmöte.

\subsubsection{Stadgar}

\paragraph{Stadgeändring}

Stadgarna ändras genom ett enhälligt beslut av årsmötet eller
två tredjedelars majoritet på två på varandra följande årsmöten, varav
minst det ena måste vara ett ordinarie enligt \S\ref{sec:ordinarie_sm}

\subsubsection{Reglemente}

\paragraph{Syfte}

Reglementet reglerar sektionsverksamheten där stadgarna ej är tillräckligt
utförliga. Reglementet är alltid underordnat stadgarna.

\paragraph{Reglementesändring}

Reglementet kan ändras genom beslut på ett styrelsemöte. Punkter som måste finnas enligt
stadgarna kan dock inte avskaffas utan stadgeändring.

\subsection{Beslutsnivåer}

Beslut fattas med enkel majoritet om inget annat är föreskrivet.

\subsection{Officiella informationskanaler}
\label{sec:officiella_informationskanaler}

Information som anslås skall spridas till sektionsmedlemmar via de informationskanaler som regleras via reglementet.

\subsection{Definitioner}

\subsubsection{Läsdag}

Definitionen av läsdagar som benämns i sektionens styrdokument är
måndag till och med fredag under terminstid, inklusive tentamensperiod, dock ej helgdagar.

\subsection{Firmatecknare}

Ordförande och kassör tecknar firman var för sig. Styrelsen kan fatta beslut
om ytterligare firmatecknare. 

\section{Medlemskap}

Sektionsmedlem är

\begin{itemize}
  \item ordinarie sektionsmedlem enligt \S\ref{sec:ordinarie_sektionsmedlem}
  \item hedersmedlem enligt \S\ref{sec:hedersmedlem}
\end{itemize}

\subsection{Ordinarie sektionsmedlem}
\label{sec:ordinarie_sektionsmedlem}

\begin{itemize}

  \item Ordinarie sektionsmedlem är de som skrivit in sig i sektionens officiella medlemsregister samt erlagt medlemsavgiften.
  \item Medlemsavgiften faställs årligen av Årsmötet

\end{itemize}

\subsubsection{Rättigheter}

Ordinarie sektionsmedlem har rätt att

\begin{itemize}
  \item deltaga med yttranderätt och rösträtt på årsmöte
  \item deltaga med yttranderätt och rösträtt på medlemsmöte
  \item få motion eller proposition behandlad av styrelsen
  \item kandidera till samtliga förtroendeuppdrag inom sektionen
  \item närvara på styrelsemöte såvida det inte beslutats om lyckta dörrar.
\end{itemize}

\subsection{Hedersmedlem}
\label{sec:hedersmedlem}

Sektionen kan utse till hedersmedlem sådan person som synnerligen främjat
sektionens intressen och strävanden. Förslag till hedersmedlem lämnas av
sektionsmedlem. Hedersmedlem utses av årsmötet med 5/6 majoritet. Faller fråga om val
av hedersmedlem införs varken förslag eller beslut i protokoll.

\subsection{Medlemsavgift}

Medlemsavgift för ordinarie medlemmar fastställs på årsmötet. Hedersmedlemmar är befriade från medlemsavgift.

\subsection{Alumnimedlem}
\label{sec:alumnimedlem}

Kanske en bra idé :)

\section{Årsmötet}

\subsection{Syfte}

Årsmötet är sektionens högsta beslutande organ.

\subsection{Sammansättning}

Vid årsmöte har samtliga ordinarie sektionsmedlemmar närvarorätt, yttranderätt,
yrkanderätt och rösträtt. Hedersmedlemmar har närvaro- och yttranderätt. Dessutom kan årsmötet
adjungera utomstående med närvarorätt och eventuellt även yttranderätt.

\subsection{Uppgifter}

Det åligger Årsmötet

\begin{itemize}
  \item att fastställa riktlinjer och budget för sektionens verksamhet
  \item att granska revisorers berättelser samt sektionens
    ekonomiska redovisning
  \item att faställa medlemsavgift för medlemmar i sektionen
  \item att ta ställning till ansvarsfrihet för föregående verksamhetsårs styrelse
  \item att välja ny styrelse

\end{itemize}

\subsection{Kallelse}
\label{sec:kallelse}

Styrelsen kallar till ordinarie och extra årsmöte.

Kallelse till ordinarie årsmöte skall anslås enligt
\S\ref{sec:officiella_informationskanaler} senast 10 läsdagar före mötet för att mötet skall anses vara
behörigt utlyst.

Kallelse till extra årsmöte skall anslås enligt
\S\ref{sec:officiella_informationskanaler} senast 10 läsdagar före mötet för att mötet skall anses vara
behörigt utlyst.

Dagordning och övriga handlingar skall anslås jämte kallelse senast
5 läsdagar före mötet.

Om minst 30 sektionsmedlemmar eller sektionsrevisor enligt \S\ref{sec:revisorer} så begär hos styrelsen, skall extra årsmöte hållas inom 20 läsdagar.

\subsection{Beslut}

Beslut kan endast fattas i fråga som antingen enligt dagordningen skall behandlas
eller berörs av proposition eller motion. Beslut fattas med enkel majoritet
såvida inget annat stadgats. Vid lika röstetal har mötesordföranden
utslagsröst, förutom vid personval då lotten avgör. Sluten omröstning skall ske
om någon röstberättigad deltagare så begär. Sektionsmedlem äger inte rätt att rösta genom ombud, utan
endast genom personlig närvaro på årsmötet.

Ingen får delta i beslut när frågan om ansvarsfrihet
för honom/henne själv behandlas eller berör personens sambo flickvän, pojkvän eller dylikt.

\subsection{Protokoll}

Vid årsmötet skall diskussionsprotokoll föras av mötessekreterare och justeras av
mötesordföranden jämte två av mötet utsedda justerare. Protokoll skall
innehålla en förteckning över närvarande och röstberättigade medlemmar. Protokoll
skall i justerat skick anslås enligt \S\ref{sec:officiella_informationskanaler}.

\subsection{Proposition och motion}

Motion eller proposition till årsmöte skall vara styrelsen tillhanda senast 10
läsdagar före årsmötet.

Styrelsen ansvarar för att motioner och proposition anslås tillsammans med dagordningen.

\subsection{Sammanträden}

Det skall förflyta minst tio läsdagar mellan två på varandra följande årsmöten. Årsmöten
får inte hållas under tentamensperiod.

\subsubsection{Ordinarie årsmöte}
\label{sec:ordinarie_sm}

Som ordinarie årsmöte räknas ej extra årsmöte. Det
skall hållas minst ett ordinarie årsmöte per läsår.

\subsubsection{Extra årsmöte}

Styrelsen kan, själv eller på anmodan, kalla till extra årsmöte. Extra årsmöte kan
endast behandla den eller de frågor som angivits i kallelsen, således behandlas
ej övriga motioner eller interpellationer. Dock kan övrig fråga
väckas.

\section{Styrelsen}

\subsection{Syfte}

Styrelsen är sektionens styrelse och högsta verkställande organ.

\subsection{Sammansättning}

Styrelsen består av

\begin{itemize}
  \item Ordförande
  \item Vice ordförande
  \item Kassör
  \item Sekreterare
  \item och de av styrelsen utsedda ledamöter
\end{itemize}

Dessa har närvaro-, yttrande-, yrkande- och rösträtt vid styrelsemöte. Sektionens
revisorer enligt har närvaro-, rösträtt, yttrande- och yrkanderätt vid styrelsemöte. 
Funktionärer har närvaro- och yttranderätt vid styrelsemöte. Övriga sektionsmedlemmar har närvarorätt
vid styrelsemöte. Därutöver äger styrelsen rätt att adjungera person med närvaro-
eller närvaro- och yttranderätt för viss fråga eller helt möte. styrelsen
äger vidare, om synnerliga skäl föreligger, rätt att besluta om lyckta dörrar,
vilket utestänger samtliga utan yrkanderätt.

\subsection{Styrelsen}
\label{sec:styrelsen}

\subsubsection{Kallelse}

Ordförande kallar till styrelsemöte. Kallelsen skall delges genom mail senast 5 läsdagar före mötet.

\subsubsection{Beslut}

Styrelsemötet är beslutsmässigt om minst hälften av dess ledamöter är närvarande, och
mötet är behörigt utlyst enligt \S\ref{sec:styrelsen} Vid lika röstetal har
mötesordförande utslagsröst.

\subsubsection{Protokoll}

På styrelsemötet skall protokoll föras. Protokollet skall justeras av mötesordföranden
jämte en av mötet utsedd justerare. Protokollet skall delges via mail till styrelsen i justerat skick senast 14 dagar efter
mötet.

\subsection{Uppgifter}

Det åligger styrelsen

\begin{itemize}
  \item att sköta sektionens löpande förvaltning
  \item att verkställa av årsmötet fattade beslut
  \item att i brådskande fall utöva årsmötets:s befogenheter. Sådant fall skall dock
    alltid prövas på nästkommande årsmöte
  \item att efter skriftlig begäran från en av årsmötes vald ledeamot entlediga
    densamme
  \item att vid behov och efter majoritetsbeslut vid styrelsemöte tillförordna
    intresserad sektionsmedlem till vakant post inom sektionen.
  \item att vid behov och efter majoritetsbeslut vid styrelsemöte utöva ordförandeskap
    för underförening i dess ordförandes ställe.
  \item att svara för att verksamhetsplan, budget, verksamhetsberättelse och
    årsbokslut upprättas
  \item att, om så anses nödvändigt, avsätta en av sektionen vald ledamot.
    En sådan avsättning skall dock alltid prövas på nästkommande årsmöte.
\end{itemize}

\subsection{Brådskande ärenden}

I brådskande fall äger styrelsen rätt att utöva årsmötets befogenheter.
styrelsen äger dock ej därigenom rätt att ändra stadgar. Beslut enligt detta stycke skall
prövas på nästföljande årsmöte.

I brådskande fall äger ordförande rätt att utöva styrelsens
befogenheter. Ordförande äger dock ej därigenom rätt att utöva årsmötets
befogenheter enligt första stycket. Beslut enligt detta stycke skall prövas på nästföljande styrelsemöte.

\subsection{Ställföreträdande ordförande}

Om ordförande är oförmögen att göra så, utövar vice ordförande
dennes befogenheter, och fullgör dennes plikter.

\subsection{Per capsulam-beslut}

Vid per capsulam beslut gäller 2/3-majoritet och att beslut prövas på
nästkommande styrelsemöte.

\subsection{Skamråd}

Styrelsen må, om det så önskar, utfärda skamråd, vilka utgöra
rekommendationer å de enskilda sektionsmedlemmarnas liv och leverne.

\section{Organisation}

\subsection{Underföreningar}
\label{sec:namnder}

\subsubsection{Syfte}

En underförening är ett officiellt sektionsorgan med syfte att ansvara för en viss del
av sektionens verksamhet. Underföreningar driver sin verksamhet självständigt inom
ramen för av årsmötet och styrelsen fattade beslut. Underföreningar är de, som upptas i
reglementet.

\subsubsection{Sammansättning och verksamhet}

En nämnds sammansättning och verksamhet regleras i reglementet.

\paragraph{Ordförande}

För varje nämnd skall det finnas en ordförande.
Nämndens ordförande är ansvarig för underföreningens verksamhet samt att dess
reglemente hålls aktuellt.

\subsubsection{Skyldigheter}

Underförening är skyldig att upprätta verksamhetsberättelse, samt även annars på
anmodan från styrelsen eller årsmöte fullständigt redovisa sin verksamhet för
densamme.

\paragraph{Urnval}

Alla styrelsemedlemmar väljs med urnval i enlighet med sektionens
reglemente.

\paragraph{Val vid extra årsmöte}

I undantagsfall kan val på extra årsmöte utföras. Valberedningen skall anslå
en nomineringslista senast 8 läsdagar före det extra årsmöte då ett val sker. På denna lista
kan sektionsmedlem nomineras till ledamotspost. Nominering till ledamotspost
måste lämnas in senast en (1) läsdag före det extra årsmöte där valet sker. Nominering till
ledamotspost måste accepteras innan öppnandet av det extra årsmötes där valet sker för
att kandidaturen ska vara giltig.

\subsubsection{Obligatoriska underföreningar}

Det skall finnas ett studieråd.

\subsection{Ledamöter}

\subsubsection{Ändamål}

Ledamot är den som av årsmöte eller vid urnval har valts till ett
förtroendeuppdrag. En ledamots verksamhet och uppdrag regleras i
reglementet.

\subsubsection{Skyldigheter}

En ledamot ansvarar för sitt verksamhetsområde samt för att ledamotens del
av reglementet hålls aktuellt. Ledamoten är skyldig att löpande hålla
styrelsen informerad om sitt verksamhetsområde, samt att på anmodan från
styrelsen eller årsmöte fullständigt redovisa sin verksamhet för densamme.

\subsubsection{Mandatperiod}

Ledamotens mandatperiod sammanfaller med verksamhetsår om inget annat är
föreskrivet i reglementet. Ordinarie val skall hållas på mandatperiodens sista
ordinarie årsmöte.

\subsubsection{Obligatoriska ledamöter}

\begin{itemize}

  \item Utöver styrelsens ledamöter 
  \item Revisorer 
  \item och ordförande för de under \S\ref{sec:namnder} uppräknade underföreningarna

\end{itemize}

\subsection{Projekt}

\subsubsection{Syfte}

Ett projekt är ett tillfälligt sektionsorgan med syfte att genomföra för
projektet avsatt ändamål. Projekt driver sin verksamhet självständigt inom
ramen för av årsmöte och styrelsen fattade beslut.

\subsubsection{Uppstart}

\paragraph{Nya projekt}

Nya projekt startas genom årsmöte-beslut. I förslag till beslut ska projektnamn,
syfte, budget, verksamhetsplan och en ungefärlig tidsplan finnas med.
Projektledare kan antingen väljas direkt av årsmötet genom fri nominering
eller genom val med samma procedur som för övriga funktionärer på
nästkommande årsmöte.

\paragraph{Återkommande projekt}

Projekt som styrelsen anser är regelbundet återkommande kan startas
utan årsmöte-beslut. De startas då istället genom beslut på styrelsemöte med en kort
motivering innehållande referens till minst ett väldigt likartat tidigare
projekt samt en uppskattning av de uppgifter, bortsett från budget, som krävs
för att starta ett nytt projekt. Val av projektledare ska ske på nästkommande
medlemsmöte enligt samma procedur som för övriga funktionärsposter. Vald projektledare
åläggs att inkomma med motion innehållandes budget för projektet till första
möjliga styrelsemöte efter valet, såvida årsmöte inte redan beslutat om budget för denna
projektomgång. Om projektledarposten vakantsätts läggs aktuell projektomgång
automatiskt ned. styrelsen ansvarar för att en förteckning över återkommande
projekt finns tillgänglig för alla sektionens medlemmar på dess hemsida.

\subsubsection{Avslutning}

Efter att projektets verksamhet är genomförd ska projektledare snarast möjligt
överlämna avslutad bokföring och verksamhetsberättelse till styrelsen.
På nästkommande styrelsemöte ska frågan om formellt avslutande av projektet tas upp.
styrelsemötet äger rätt att besluta om avslutande av projekt även om projektledare inte
överlämnar bokföring och verksamhetsberättelse inom skälig tid.

\subsubsection{Projektledare}

För varje projekt skall det finnas en eller flera personer som är
projektledare. Som projektledare kan bara räknas sektionsmedlem som tillsatts
av medlemsmöte eller årsmöte. Projektledare kan formellt ha en annan titel såsom direktör, marskalk,
general eller liknande om denna titel har godkänts av årsmöte eller styrelsemöte.

\paragraph{Skyldigheter}

Projektledare är ansvarig för projektets verksamhet, ekonomi, bokföring samt
val av projektmedlemmar om inte annat beslutas av årsmöte. Projektledare är
skyldig att på anmodan från styrelse eller årsmöte fullständigt redovisa
projektets verksamhet och ekonomi för densamme.

\paragraph{Rättigheter}

Styrelsen ansvarar för att projektledare i aktiva projekt får ta del av
samma information som sektionens funktionärer samt att även dessa bjuds
in till funktionärsmiddagar och liknande tillställningar. Styrelsen äger
rätt att fritt bedöma vilka projekt som anses vara aktiva.

\section{Revision}

\subsection{Revisorer}
\label{sec:revisorer}

Årsmötet skall utse två revisorer.

\subsubsection{Befogenheter}

Revisorerna har rätt

\begin{itemize}
  \item att närhelst de så önskar ta del av samtliga räkenskaper, protokoll och
    andra handlingar
  \item att begära och erhålla upplysningar rörande verksamhet och förvaltning
  \item att bevaka samtliga sektionsorgans sammanträden med yttrande och
    yrkanderätt
  \item att kalla till möte med samtliga sektionsorgan.
\end{itemize}

\subsubsection{Uppgifter}

Det åligger revisorerna

\begin{itemize}
  \item att fortlöpande granska sektionens förvaltning och verksamhet
  \item att senast 5 läsdagar före de årsmöten vid vilka fråga om ansvarsfrihet
    behandlas inlämna revisionberättelse
    till styrelsen.
\end{itemize}

\subsection{Verksamhetsberättelse och årsbokslut}

Sektionens verksamhetsberättelse och årsbokslut skall överlämnas till
revisorerna senast 15 läsdagar före det årsmöte på vilka de skall granskas, samt
anslås.

\section{Upplösning}

\subsection{Beslut}

Upplösning av föreningen kan endast ske vid årsmöte, där minst 2/3 av rösterna är för förslaget.

\subsection{Avregistrering}

Ordförande och kassör ansvarar för avregistrering av föreningen.

\subsection{Tillgångar}

I händelse av upplösning skall föreningens tillgångar användas till att köpa in gravöl till styrelsen, och/eller skänkas till något fysikfrämjande projekt.

\end{document}
