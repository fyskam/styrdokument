\documentclass{../resources/dgovdoc}


\usepackage[utf8]{inputenc}

\usepackage{hyperref}

\title{Jämlikhetspolicy}

\begin{document}

\maketitle

\section{Inledning}

Denna policy gäller Fysikums Kamratförening (FysKam) inklusive alla
föreningens underföreningar, nämder och projekt. Dessa hänvisas fortsättningsvis
med beteckningen FysKam. Denna policy ska följas upp och revideras
varje verksamhetsår för att sedan fastställas av FysKams årsmöte.

\section{Grundläggande principer}

FysKam bedriver en omfattande verksamhet, både i form av
utbildnings- och studiesocial bevakning, samt olika studentikosa och sociala aktiviteter,
där t.ex. mottagning av nya studenter spelar en stor roll. Det är viktigt att all
verksamhet bedrivs på ett sådant sätt att alla är och känner sig välkomna och blir
jämlikt behandlade. Arbetet kring jämlikhet och jämställdhet genomsyrar FysKams
verksamheter och arrangemang.

\subsection{FysKam ska verka för att}

\begin{itemize}
\item verksamheten präglas av ömsesidig respekt, medmänsklighet och jämlikhet
\item verksamheten i möjligaste mån planeras på ett sådant sätt att alla kan delta
\item en medvetenhet ska finnas inom föreningen om vilka attityder och
värderingar FysKam som organisation, och representanter för denna, förmedlar till
medlemmarna och gentemot allmänheten
\item följa upp kränkande eller diskriminerande händelser och förhindra att liknande
situationer uppstår igen
\item genom attitydförändrande, arbete stävja diskriminering, trakasserier och
anvädningen av härskarteknik inom den egna föreningen
\end{itemize}

\subsection{FysKam accepterar inte en verksamhet där}

\begin{itemize}
\item någon individ eller grupp av individer särbehandlas positivt eller negativt i
relation till andra på grund av kön, etnisk tillhörighet, religion eller annan
trosuppfattning, sexuell läggning, könsöverskridande identitet eller uttryck,
funktionshinder, ålder, politisk uppfattning, social bakgrund eller val av
utbildning
\item trakasserier, diskriminering eller andra kränkande moment förekommer.
Huruvida en händelse eller miljö är att betrakta som kränkande eller inte
bedöms utifrån den utsatta partens upplevelse av situationen
\item någon individ eller grupp av individer kategoriskt förutsätts tillhöra en bestämd
etnisk-, social-, religiös-, eller ideologisktillhörighet, sexuell läggning, social
bakgrund eller utbildning eller könsöverskridande identitet eller uttryck
\end{itemize}

\section{Ansvarsordning}

Ansvarsordningen avser information om och innehållet i jämlikhetspolicyn samt att
policyn efterlevs.
FysKams styrelse och studiebevakare med studiesocialt ansvar, ansvarar för att
jämlikhetsfrågor diskuteras inom FysKams olika grupper, samt att dessa för information
om denna policy.

\begin{enumerate}
\item FysKams styrelse
\item Studiebevakare med studiesocialt ansvar
\item Ansvarig för verksamheten/arrangemanget
\end{enumerate}


\end{document}
