\documentclass{../resources/dgovdoc}


\usepackage[utf8]{inputenc}

\usepackage{hyperref}

\title{Reglemente}

\begin{document}

\maketitle

\section{Insignia}

\subsection{Färg}

Sektionens färg är mörkblå

\subsection{Symbol}

Lämplig beskrivning av märket passar bra här

\section{Sektionsklädsel}

Sektionens nuvarande officiella sektionsklädsel är Skamrocken, men även äldre sektionsklädslar såsom Skamkappan och FysKams overall
räknas som sektionsklädsel.

\subsection{Färg}

Skamrocken skall vara i samma färg som sektionens färg.

\subsection{Rekomenderat utseende och placering av diverse märken}

Läsaren hänvisas till bilaga 1

\section{Officiella informationskanaler}

Information som anslås skall sättas upp i pappersform och/eller skrivas upp på tavlan i sektionslokalen samt göras tillgänglig på sektionens officiella webbplats.

\subsection{Inofficiella informationskanaler}

Andra informationskanaler som med fördel kan tas i bruk, är:

\begin{itemize}

\item Facebook
\item Twitter
\item Anslagstavlor
\item Utropare
\item Flygblad
\item Gatukritor

\end{itemize}

\section{Underföreningar}

Det åligger samtliga underföreningar, nämder och funktionärer att följa sektionens ekonomiska styrdokument,
FysKam:s alkoholpolicy och FysKam:s jämlikhetspolicy. 

\subsection{Studierådet}

\subsubsection{Syfte}

Studierådets syfte är att bevaka och förbättre utbildningskvaliteten och studiemiljön för sektionens medlemmar på såväl kort som lång sikt.

\subsubsection{Organisation}

\subsubsection{Verksamhet}

\subsection{FUSK - FysKams Utilitäristiska SpelKompani}

\subsubsection{Syfte}

\subsubsection{Organisation}

\subsubsection{Verksamhet}

\subsection{Festkommiten}

\subsubsection{Syfte}

Festkommiten anordnar fester och andra sociala arrangemang för sektionens medlemmar och i vissa fall även deras eventuella vänner. Att främja frändskap mellan sektionens medlemmar och stifta gemenskap över sektionsgränserna. 

\subsubsection{Organisation}

Festkommiten leds av Festansvarig. Övriga medlemmar utses av Festkommiten. 

\subsubsection{Verksamhet}

Det åligger Festkommiten att

\begin{itemize}
\item arrangera fester, jippon, högtidliga ceremonier och andra sociala arrangemang
\item vid behov assistera vid andra sektionsrelaterade arrangemang
\item skicka ut inbjudningar till övriga sektioner och andra högskolor till av Festkommiten arrangerade evenemang
\item i sektionens informationskanaler informera om till Festkommiten inkomna inbjudningar till externa evenemang, samt fördela platser och biljetter om antalat aspiranter överstiger platsantalet. 
\item skicka ut skriftlig inbjudan till Kamratskapen till medlemmar av ordern, Ordförande Emereitus. 
\item i största möjliga mån sträva efter annordnadet av spårvagnsrelaterade aktiviteter
\end{itemize}

\subsubsection{Bokföringsplikt}

Festkommiten är bokföringspliktigt. 

\section{Förtroendeposter}

\subsection{Styrelsen}

\subsubsection{Ordförande}

\begin{itemize}
\item Är ordförande för sektionen och leder styrelsen arbete.
\item Kallar till möten och leder i regel dessa.
\item Bistår kassören i arbetet med ekonomin.
\item Sitter i UTNs ordföranderåd och är kontaktperson för föreningen.
\item Om vakant faller posten på Vice ordförande
\end{itemize} 
Ordförande bör ej åta sig andra tidskrävande uppdrag under sin mandatperiod. 
Väljs på årsmöte. Har verksamhetsår som mandatperiod.

\subsubsection{Vice ordförande} 

\begin{itemize}
\item Bistår i huvudsak ordförande men också övriga i styrelsen i dess arbete.
\item Är suppleant i FUM
\item Har ansvar för kontakten med underföreningar, nämnder och projekt, som saknar annan kontaktperson inom styrelsen.
\item Om vakant fördelas postens ansvar bland styrelsen.
\end{itemize}
Väljs på årsmöte. Har verksamhetsår som mandatperiod. 

\subsubsection{Sekreterare}

\begin{itemize}
\item Är styrelsens sekreterare
\item Är i regel sekreterare på styrelse- och medlemsmöten
\item Hämtar och delar ut sektionens post.
\item Om vakant faller posten på Vice ordförande
\end{itemize}
Väljs på årsmöte. Har kalenderår som mandatperiod. 

\subsubsection{Kassör}

\begin{itemize}
\item Håller reda på och sköter föreningens ekonomi, och är ytterst ansvariga för den. 
\item Planerar budget, sköter löpande bokföring samt placerar sektionens tillgångar. 
\item Bistår med sektionens ekonomiska beslutsunderlag. 
\item Kassören ansvarar även för att det finns ett uppdaterat styrdokument för sektionens ekonomi. 
\item Representant i UTN:s ekonomiråd
\item Om vakant faller posten på Ordförande
\end{itemize}
Väljs på årsmöte. Har verksamhetsår som mandatperiod. 

\subsubsection{Inköpsansvarig}

\begin{itemize}
\item Ansvarar för prissättning och inhandlande av varor till lokalen. 
\item Sköter all kontakt med grossister och leverantörer
\item Bokför alla inköp till lokalen så att man lätt kan se en förväntad vinst
\item Om vakant faller posten på Vice ordförande
\end{itemize}

\subsubsection{Festansvarig}

\begin{itemize}
	\item Är sektionens klubbmästare tillika ordförande för Festkommiten.
\end{itemize}

\subsubsection{Informationsansvarig}

\begin{itemize}
	\item Representant i UTN:s informationsråd
	\item Sköter sektionens årliga medlemsgallring
\end{itemize}

\subsubsection{Lokalansvarig}

\subsubsection{Alumnansvarig}

\subsection{Underföreningsordförande}

\subsection{Övriga förtroendeposter}

\subsubsection{Kårfullmäktigeledamöter}

\paragraph{Ändamål}

Kårfullmäktigeledamöter och -suppleanter representerar sektionen i UTN:s kårfullmäktige

\paragraph{Organisation}

Sektionen bör ha en ledamot och en suppleant för varje mandat i Kårfullmäktige som sektionen tilldelas.

\paragraph{Verksamhet}

Såväl ledamöter som suppleanter skall delta på så många sammanträden av kårfullmäktige som möjligt.
De är solidariskt ansvariga för att sektionen är fulltalig vid samtliga sammanträden. 
Ledamöter och suppleanter skall rösta så, som de själva finner lämpligast. Dock skal de i största möjliga mån
bevaka sektionens och sektionsmedlemmarnas intressen. 

\paragraph{Vakans}

Om vakant faller ansvaret på Ordförande

\subsubsection{Jämlikhetsansvarig}

\paragraph{Syfte}

Att värna och upplysa om jämlikhet och mångfald på sektionen.

\paragraph{Verksamhet}

\begin{itemize}

\item Gör medlemmar medvetna om dess rättigheter och vart de skall vända sig om de känner sig trackaserade eller kränkta.
\item Arbeta för ökad jämlikhet...
\item Ansvarar för att FysKams jämlikhetspolicy hålls uppdaterade

\end{itemize}

\subsubsection{Fanbärare}

Fanbärarna försvarar sektionens ära genom att bära dess fana vid olika högtidliga tillfällen.
Observera att fanan skall hållas högt. Att vara Fanbärare är en mycket hedersfylld post på sektionen. 
Fanbärarna skall närvara på så många som möjligt av de tillställningar som sektionen blir inbjudna till.
Fanbärarna bär huvudansvaret för att sektionens fana hålls i gott skick. 
Väljs på årsmötet. Har verksamhetsår som mandatperiod. Har företräda framföra vanliga medlemmar på
tillställningar där sektionen är inbjuden.

\subsubsection{Vice fanbärare}

Vice fanbärare försvarar sektionens ära när ordinarie fanbärare ej har möjlighet att göra det. 
Vid arrangemang med begränsat deltagarantal har fanbäraren företräde framför vice fanbäraren

\subsubsection{Sektionhistoriker}

Sektionshistorikern skall se till att sektionens ärorika historia inte faller i glömska, dels genom att samla in historisk information och historiska föremål och dels genom att föra in sagda information vidare till och visa upp sagda föremål för sektionsmedlemmarna i lämpliga sammanhang. 
Sektionshistorikern ansvarar även för sektionens alumniverksamhet. 
Sektionshistorikern avgör själv hur hen bäst uppfyller ändamålet. Väljs på årsmöte, har kalenderår som mandatperiod. 

\subsubsection{Revisorer}

\paragraph{Syfte}

Revisorernas uppgift är att övervaka styrelsens och underföreningarnas arbeten.

\paragraph{Organisation}

Enligt sektionens stadgar finns det två revisorer, utsedda av årsmötet. De skall:

\begin{itemize}
\item Övervaka styrelsen och underföreningarnas arbete i sektionens namn
\item Övervaka den löpande bokföringen och, om så anses nödvändigt, kräva att en delårsrapport presenteras
\item Revidera ekonomisk bokföring från sektionens organ
\item Övervaka upprättandet av verksamhetsberättelsen för sektionen samt 
\item Vara skiljemän vid tvister inom sektionen där parter inte behöver använda sig av årsmötet
\end{itemize}

Tvister där sektionens revisorer inte kan vara skiljemän inkluderar,
men är inte begränsat till tvister där revisorerna kan anses vara jäviga. 

\paragraph{Verksamhet}

Revisorerna för ett verksamhetsår är ålagda att reviderar samtliga av sektionens verksamheter
för det året, samt att i samråd med tidigare och senare revisorer reviderar löpande verksamhet som löper över flera år.
Det åligger senast valda revisorerna att ansvara för att revisionerna genomförs.

\subparagraph{Revisionsberättelse och -rapport} 

Varje revision dokumenteras i två skrivelser. Av dessa två är revisonsberättelsen offentlig. 
Revisionsrapporten är en detaljerad beskrivning av anmärkningar i bokföring och/eller verksamhet.
Den ligger till grund för kommunikationen mellan sektionens revisorer från år till år. I revisionsrapporten bör antecknas:

\begin{itemize}
\item anmärkningar på bokföringens genomförande och strukturering
\item händelser i verksamheten som påverkat andra organ av sektionen, samt
\item revisorernas uppfattning om verksamheten givet verksamhetsberättelse och samtal med av årsmötet utnämnda underföreningsansvariga.
\end{itemize}

Revisionsberättelsen baseras på revisionsrapporten och är det dokument som presenteras för årsmötet vid fråga angående ansvarsfrihet. Revisionsberättelsen är en kort sammanfattning av rapporten, med avslutande rekommendation att tillstyrka eller avstyra beviljande av ansvarsfrihet. Rekommendationen kan utelämnas då särskilda skäl föreligger det emot.

\subparagraph{Årsmöte}

Vid ett årsmöte där en revisionsberättelse skall läsas, kan revisorerna, enligt föregående avsnitt, ge en rekommendation till årsmötet angående beviljande av ansvarsfrihet. Årsmötet bör beakta revisorernas samlade arbete vid efterföljande omröstning.

\subparagraph{Normativa rekommendationer}

De rekommendationer som ges nedan bör följas för att förenkla och accelerera revisionsförfarandet

\subparagraph{Verksamhetsberättelse}

Det åligger Ordförande att ansvara för att en verksamhetsberättelse (VB) uppförs efter (eller i samband med) avslutat verksamhetsår. Denna VB skall (som ett minimum) innehålla en berättelse från varje underföreningsordförande. VB skall vara revisorerna tillhanda innan första årsmötet på nästkommande verksamhetsår.

\subparagraph{Bokföring}

Det åligger kassören och de ekonomiskt ansvariga i varje underförening att lämna en avslutad bokföring till revisorerna. Bokföringen skall vara revisorerna tillhanda första årsmötet på nästkommanda verksamhetsår, om inte starka skäl föreligger däremot.
Det åligger även de ekonomiskt ansvariga att på ett professionellt och strukturerat sätt inventera lager och kassa vid överlämnandet till nästa förtroendevald på posterna. Överlämningsdokumentet skall finnas revisorerna tillhanda tillsammans med bokföringen. 

\subparagraph{Mandatperiod}

Revisor väljs på årsmötet till sakrevisor för sektionen under ett verksamhetsår.

\subsection{Valberedning}

\begin{itemize}

\item Varje förtroendevald ansvarar för att valbereda sin egen post
\item Valberedningen ska uppmuntra till sektionsengagemang och bistå med information och vägledning
\item 

\end{itemize}

\section{Ordinarie årsmöte}

\subsection{Dagordning}

Här följer ett bra exempel på hur dagordning kan se ut på ett årsmöte

\begin{itemize}
\item punkter som beskriver hur ett bra årsmöte förs
\end{itemize}

\section{Förtjänsttecken och ordnar}

\subsection{Medaljen}

Sektionens finaste förtjänsttecken heter Medaljen och utgörs av en medalj. 

\subsubsection{Syfte}

Medaljen utdelas till de sektionsmedlemmar som synnerligen förtjänstfullt verkat ideellt för sektionen. 

\subsubsection{Förslagslämning}

Sektionsmedlem kan när som helst inlämmna förslag på mottagare av Medaljen, med motivering till styrelsen. 

\subsubsection{Utdelning}

Styrelsen utnämner mottagare av Medaljen, vilka presenteras vid årsmötet. Utdelning av Medaljen sker på kamratskapen, eller motsvarande högtidligt tillfälle samma år. 

\subsection{Ordnar}

Sektionen har två ordnar benämda ``Orförande Emiritus'' och ``Festansvarig Emeritus''.

\subsubsection{Ordförande Emeritus}

Ordförande Emeritus tilldelas de sektionsordförande som förtjänstefullt arbetat under en hel mandatperiod. 
    Vidare gäller att Ordförande Emeriti
\begin{itemize}
\item erhåller evigt kostandsfritt medlemskap i sektionen som Hedersmedlem
\item erhåller årlig speciell inbjudan till Kamratskapen
\end{itemize}

\subsubsection{Festansvarig Emeritus}

Festansvarig Emeritus tilldelas de Festansvariga som förtjänstefullt arbetat under en hel mandatperiod.

\subsection{Förtroendevaldsmedalj}

\subsubsection{Syfte}

Förtroendevaldsmedalj utdelas till de sektionsmedlemmar som förtjänstefullt under en hel mandatperiod tjenstegjort som funktionär på sektionen. 

\subsubsection{Utdelning}

Endast en medalj per post och mandatperiod. Styrelsen ansvarar för att medaljen utdelas på kamratskapen eller motsvarande högtidligt tillfälle

\subsection{Projektledarmedaljen}

\subsubsection{Syfte}

Projektledarmedaljen utdelas till de sektonsmedlemmar som förtjänstefullt planerat och genomfört ett projekt vid sektionen i egenskap av projektledare samt i förekommande fall fullständigt avslutat den ekonomiska bokföringen. 

\subsubsection{Urval}

För bedömmning av hurvida en projektledare arbetat förtjänstefullt ansvarar styrelsen. Vid denna bedömmning bör särskild vikt läggas vid att projektet tillför något för sektionens medlemmar samt att det ekonomiska resultatet inte med marginal understiger av sektionen godkänd budget. 

\subsubsection{Utdelning}

En medalj per person och projekt utdelas. Styrelsen ansvarar för att utdelning sker på kamratskapen eller motsvarande högtidligt tillfälle.

\section{Sektionslokalen}

\subsection{Alkohol}
Kanske någonting här vad vet jag

\subsection{Övriga regler}

Det får max vistas 200 personer i lokalen. 

\section{Övrigt}

\subsection{Hedersbetygelser}

Ordförande bör hälsas med honnör när denne inträder i sektionlokalen

\subsubsection{Typ av honnör}

Ordförande ser helst att honnören är av fransk-brittisk typ.

\end{document}
