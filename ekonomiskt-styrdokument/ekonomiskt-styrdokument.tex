\documentclass{dgovdoc}


\usepackage[utf8]{inputenc}

\usepackage{hyperref}

\title{Ekonomiskt Styrdokument}

\begin{document}

\maketitle

\section{Bakgrund}

Detta dokument ska fylla två funktioner, dels ska det ange de för sektionen gällande regler för hur ekonomin sköts. Dokumentet ska även fungera som en snabb utbildning i hur förtroendevalda, nämnder och underföreningar ska sköta sin ekonomi och ekonomiska redovisning. 

\section{Bokföringsskyldiga nämnder samt samtliga projekt och sektionens kassör}

Bokföringspliktiga underföreningar och nämnder såsom Festkommiten, FUSK, samt projekt sköter sin egen bokföring. Bokföringen skall skötas löpande och vara färdig senast nästföljande månadsskiftet. Alla transaktioner på bankkonto ska bokföras av respektive underförening.

\section{Skulder till engagerade}

Kvitton för inköp åt sektionen skall senast lämnas in 30 dagar efter då inköpet gjordes för att skulden ska betalas tillbaka. Skulder ska betalas tillbaka till engagerade inom 5 veckor under förutsättning att alla uppgifter är rätt registrerade och det framgår i vilket syfte inköpet är gjort. 

\section{Hantering av kontanter}

Festkommiten är dagsläget den enda nämnden som har tillräckliga rutiner för att ha en egen handkassa, andra nämnder eller underförening ska därför låta FysKam hantera deras kontanter och sedan lösa eventuella skulder genom överföringar til/från FysKams bankkonto.

\section{Fakturor}

En engagerad medlem får handla mot faktura i sektionens namn endast om det finns en budgetpost för inköpet och personen har ett godkännande av respektive nämdordförande eller styrelsen. Mottagna fakturor skall sättas in i kvittopärmen. Om nämnden själv betalar fakturan skall detta samt fakturans betalningsdag markeras på fakturan. 

\section{Fakturering}

Underförening ansvarar själva för att kontrollera att utskickade fakturor betalas i tid. En kopia av fakturan skall omgående sättas in kvittopärmen. 

\end{document}
